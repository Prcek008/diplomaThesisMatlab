\chapter*{Abstrakt}
\label{abstract:czech}

Diplomová práce se věnuje problematice hodnocení kvality zvuku ve vztahu k nastavení parametrů vysílání v systémech digitálního rádia. Nejprve jsou stručně popsány systémy DAB/DAB+, DRM(+) a techniky zdrojového kódování zvuku. Další část se věnuje postupům hodnocení poslechové kvality. Napřed je uveden přehled subjektivních poslechových metod a poté jsou vysvětleny principy objektivního testování různými počítačovými algoritmy. Praktická část se soustředí na realizaci poslechových testů MUSHRA a objektivního testování užitím čtyř různých metrik: PEAQ Basic, PEAQ Advanced, PEMO-Q a ViSQOL. Pomocí výsledků subjektivního testování je určena nejlepší metoda, která je následně použita k vyhodnocení kvality přenášeného zvukového signálu v závislosti na bitovém toku a způsobu komprese.

\bigskip

\noindent\textbf{Keywords:}

~digitální rádio, hodnocení kvality zvuku, MUSHRA, PEAQ, PEMO-Q, ViSQOL.

\vfill
