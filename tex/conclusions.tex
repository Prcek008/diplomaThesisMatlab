\widowpenalty=9999
\chapter{Závěr}
\label{chap:conclusion}

\epigraph{\textit{\hfill Nejjednodušší řešení je obvykle to správné.}}{-jedna z interpretací filozofického přístupu známého jako Occamova Břitva}

Tato práce měla několik dílčích cílů. Zpočátku bylo třeba zmapovat metody pro hodnocení zvukového signálu. Dále bylo třeba realizovat a vyhodnotit poslechový test na množině vzorků reprezentující obsah a kompresní metody používané v systémech \textit{Digital Radio Mondiale} a \textit{Digital Audio Broadcasting}. Stejnou množinu vzorků bylo potřeba oznámkovat dostupnými algoritmy pro objektivní hodnocení kvality zvuku a s ohledem na výsledek subjektivních testů vybrat metodu vhodnou pro objektivní hodnocení signálu používaného v digitálních rádiích. Posledním dílčím úkolem bylo za pomoci vybraného hodnotícího postupu posoudit kvalitu zvukových nahrávek v závislosti na bitovém toku a typu použité komprese.

Jelikož digitální rádio je středobodem této práce, jsou mu společně s popisem zdrojového kódování věnovány první dvě kapitoly. Popsány jsou možnosti a principy systémů DAB/DAB+ a DRM(+). Dále je uveden stručný vhled do problematiky kompresních standardů MPEG Layer II, MPEG 4 AAC v profilech AAC LC, HE-AAC v1 a HE-AAC v2, MPEG D xHE-AAC a kodeku Opus, který v systémech digitálního rádia nefiguruje, nicméně sdílí s xHE-AAC sjednocené kódování zvuku a řeči a je tak zajímavým doplňkem.

Konzumentem rozhlasového obsahu byl, je a pravděpodobně vždy bude člověk. Lidský vjem je tak stále nejlepším způsobem jak hodnotit kvalitu zvuku. Proto je určování kvality pomocí poslechu lidskými subjekty věnována celá kapitola. Popsány jsou postupy definované mezinárodní telekomunikační unií od přístupů obecných, přes způsob určování nepatrných zhoršení až po metodu pro střední kvalitu s označením MUSHRA.

Ta je využita při subjektivním testování. Pro určení kvality vzorků, rozdělených do kategorií: hudební, řečové s smíšené, byly navrženy dva mírně se překrývající testy. Jeden sledující kvalitu na vyšších bitových rychlostech starších kodeků, a druhý obsahující materiál zakódovaný moderními kodeky, jenž cílí na velmi nízké bitové toky. Bylo pro ně vyvinuto grafické uživatelské rozhraní v programovacím jazyce \matlab a celkově sedmnáct subjektů absolvovalo oba testy.

Pro generování vzorků, jejich objektivní testování a zobrazování získaných výsledků byla vyvinuta skupina znovupoužitelných skriptů a aplikací, taktéž v prostředí \matlab. Pomocí algoritmů PEAQ Basic, PEAQ Advanced, PEMO-Q a ViSQOL byly ohodnoceny vzorky zakódované všemi již zmíněnými kodeky.

Porovnání subjektivního a objektivního hodnocení odhalilo, že ne vždy lze použít Occamovu břitvu. Žádná z metod totiž není stoprocentně spolehlivá. Nejstarší PEAQ nejlépe hodnotí dnes už historický kodek MPEG Layer II a u novějších kompresních algoritmů klesá věrohodnost jeho hodnocení. PEMO-Q, které algoritmus PEAQ doplňuje o komplexnější model ucha sdílí podobný trend. Nejmodernější ViSQOL, původně vytvořený pro hodnocení řeči v telekomunikacích, si vede lépe u moderních metod komprese. Jako jediný se přibližuje výsledku subjektivního hodnocení vzorků zakódovaných metodou xHE-AAC. Nejstarší MPEG Layer II na nízkém bitovém toku ovšem značně nadhodnocuje. Protože jsou ale starší kodeky vytlačovány novými efektivnějšími kompresními přístupy, jeví se ViSQOL jako nejlepší volba do budoucna.

Zároveň je ale možné, že snížením váhy přikládané výstupům filtrů na vysokých kmitočtech v modelu sluchové cesty u algoritmů PEAQ či PEMO-Q by mohlo vést ke zvýšení věrohodnosti jimi podaných výsledků. Jelikož implementace algoritmů PEAQ Basic, PEMO-Q i ViSQOL byly k dispozici ve formě funkcí pro \matlab, nabízí se do budoucna možnost jejich úprav či propojení, tak aby podávaly věrohodnější hodnocení pro kompletní škálu kompresních metod.

V této práci je za prozatím nejvěrohodnější metodou zvolen ViSQOL. Pomocí něho je ohodnocen kompletní set vzorků, podrobně popsaný v kapitole \ref{chap:realization}. Výsledky jsou s komentářem k dispozici v grafické podobě v poslední kapitole této práce, věnované kvalitě přenosu zvuku v digitálních rádiích. Vynikající poslechové kvality lze při zdrojovém kódování dosáhnout pouze použitím starších kodeků na vyšších bitových rychlostech. Vývoj moderních kodeků se soustřeďuje na zvyšování kódového zisku při zachování dobré, či alespoň přijatelné kvality. Té lze dosáhnout už při datovém toku 8 kb/s pro přenos mluveného slova nebo 12 kb/s pro vysílání hudebních záznamů. Protože jsou vysílací mechanismy digitálního rádia podstatně odolnější vůči analogovému rušení, existuje velmi vysoká pravděpodobnost, že se ke koncovému zákazníkovi dostane zvukový signál ve stejné kvalitě, do jaké byl kompresní metodou zakódován. Ta tak ve výsledku mnohdy předčí kvalitu analogového rádia.