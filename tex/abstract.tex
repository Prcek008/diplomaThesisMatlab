\chapter*{Abstract}
\label{abstract:english}

The thesis deals with the issue of sound quality evaluation in relation to the setting of broadcasting parameters in digital radio systems. In the beginning, DAB/DAB+, DRM(+) systems and audio source coding techniques are briefly described. The next part focuses on the procedures of listening quality assessment. First, an overview of subjective listening methods is given, and then the principles of objective testing by various computer algorithms are explained. The practical part focuses on the implementation of listening MUSHRA tests and objective testing using four different metrics: PEAQ Basic, PEAQ Advanced, PEMO-Q and ViSQOL. Using subjective testing results the best method is chosen and used to evaluate the quality of the transmitted audio signal, depending on bitrate and compression method.

\bigskip

\noindent\textbf{Keywords:}

~digital radio, sound quality assessment, MUSHRA, PEAQ, PEMO-Q, ViSQOL.

\vfill
