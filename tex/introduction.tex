\chapter{Úvod}
\label{chap:introduction}

\epigraph{\hfill\textit{Zkuste to bez drátů, milý Marconi!}}{-Ladislav Smoljak, Zdeněk Svěrák, \textit{Jára Cimrman ležící, spící}}

%\section{Letmý pohled do historie vysílání}
Letmým pohledem do historie můžeme zjistit, že rozhlasové vysílání je tu s námi už zhruba jedno století. V roce 1922 bylo v Anglii zahájeno první pravidelné rozhlasové vysílání BBC (\textit{British Broadcasting Corporation}). Jen rok na to následoval počátek vysílání i na území Československa. Tehdejší Český rozhlas Radiojournal vysílal monofonně amplitudovou modulací a ta přenášela převážně mluvené slovo skrz éter až do roku 1959, kdy se k ní připojila modulace frekvenční.\cite{web:cro} Postupem času se k rádiím připojily i další technologie jako například televizní vysílání, telekomunikační či datové služby a elektromagnetické spektrum začalo být velmi ceněnou komoditou.

%\section{Důvody k digitalizaci}
Snaha o rozšíření nabízeného obsahu a zároveň o minimalizaci využitého pásma vedla k myšlence digitalizace rozhlasového vysílání. Digitální rádio, přenášející zvuk ve formě jedniček a nul, umožnilo využít kompresních algoritmů a tím výrazně snížit spektrální náročnost. Tento krok ovšem postavil před poskytovatele vysílání nelehký úkol. Jaké parametry přenosu nastavit, aby posluchač nepostřehl vliv digitální komprese. Jinými slovy jak zachovat dostatečnou kvalitu zdrojového signálu a zároveň docílit co nejefektivnějšího přenosu. Řešení tohoto problému se nalézalo ve výsledcích poslechových testů. Aby byly výsledky testů relevantní, bylo třeba zaručit různorodost vzorků, dostatečné množství subjektů a správnou metodiku. Staré rčení ovšem říká \uv{Čas jsou peníze} a nejinak tomu je i v případě subjektivních testů. 


Rozvoj výpočetní techniky ovšem umožnil nahradit skupiny lidí v poslechových místnostech algoritmy, odhadujícími kvalitu zvuku zakódovaného ztrátovou kompresí. Ta na rozdíl od komprese bezztrátové, nesnižuje pouze entropii dat, nýbrž z původního zdroje odstraňuje irelevanci. Jaká data mohou být označena za irelevantní pro poslech lidským uchem zkoumá ve světe zvukového zpracování obor zvaný psychoakustika. Kodeky využívají objevů této vědní disciplíny, jimiž jsou například spektrální či časové maskování, ve prospěch úspory dat. Obecně se dá říci, že čím modernější kodek je, tím více těchto poznatků využívá. To je sice velice pozitivní z hlediska přenosu signálů skrze kanál, nicméně se ukazuje, že algoritmy určené pro objektivní hodnocení audiosignálů si s takovýmto vývojem neumí dobře poradit a výsledky hodnocení dopadají při použití nových efektivnějších kodeků hůře, než kdyby byl hodnotícím prvkem lidský posluchač. To ve výsledku může vést k nesprávnému nastavení parametrů přenosu v digitálním rádiu a tím k neefektivnímu využití spektra. 

Ověřením zda k tomuto jevu dochází a v jaké míře se zabývá tato diplomová práce. V následujících kapitolách je uveden vhled do systémů digitálních rádií a v nich používaného zdrojového kódování. Dále je zde přiblížena problematika subjektivního a objektivního hodnocení audiosignálů. Podrobně jsou popsány jednotlivé metody, jenž jsou poté v praktické části využité k testování zvukových vzorků zpracovaných různými kompresními metodami. Dále je v práci popsáno, jaké programové vybavení bylo vyvinuto a využito a v závěru se práce věnuje výsledkům jednotlivých testů společně s jejich souvislostmi. Na jejich základě je vybrána a použita nejvhodnější metoda k vyhodnocování kvality zvuku v systémech digitálního rozhlasu.
